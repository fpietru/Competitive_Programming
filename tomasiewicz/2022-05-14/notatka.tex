\documentclass[10pt]{article}
\usepackage[utf8]{inputenc}
\usepackage{polski}
\usepackage{hyperref}
\usepackage{listings}
\usepackage{amsmath}

\title{14.05.2022 Notatka}
\author{Franek Pietrusiak}

\begin{document}
\maketitle

\section{\href{https://codeforces.com/problemset/problem/463/D}{D. Gargari and Permutations}}
\subsection{Problem}
Mamy $k$, $k \in \langle 2, 5 \rangle$, permutacji $n$-elementowych, $n \in \langle 1, 1000 \rangle$. Musimy zwrócić \textbf{długość najdłuższego wspólnego podciągu} dla wszystkich $k$ permutacji.

\subsection{Rozwiązanie}
Możemy potraktować problem grafowo. Wierzchołkami będą wszystkie elementy od 1 do $n$. Dwa wierzchołki (nazwijmy je $v$ oraz $u$) będą połączone krawędzią skierowaną $v \to u$, gdy element $v$ jest "przed" elementem $u$ we wszystkich permutacjach.

Teraz nasz problem sprowadza się do znalezienia \textbf{najdłuższej ścieżki w grafie}. W ogólności jest to trudne zadanie. Należy jednak zauważyć, że nasz graf nie posiada cykli oraz jest skierowany. Jest to więc \emph{skierowany graf acykliczny}. W przypadku tego rodzaju grafu przydatna staje się technika programowania dynamicznego. Niech $DP[v]$ oznacza najdłuższą ścieżkę kończącą się w wierzchołku $v$. Teraz wystarczy rozpatrywać wierzchołki w kolejności topologicznej i aktualizować ich wartość $DP$:
$$DP[v] = M + 1$$, gdzie $M$ oznacza maksymalną wartość $DP$ z wszystkich wierzchołków, z których wychodzi krawędź do $v$. Gdy nie ma takich wierzchołków $M = 0$.

Ostatecznie odpowiedzią do zadania będzie maksymalna wartość $DP$ z wszystkich wierzchołków. Złożoność: $O(n^2 k)$.

\section{\href{https://codeforces.com/problemset/problem/1486/D}{D. Max Median}}
\subsection{Problem}
Mamy $n$ elementowy ciąg elementów oraz liczbę $k$. Musimy znaleźć takie $M$, które jest największą możliwą medianą w pewnym fragmencie tego ciągu. Długość fragmentu musi być równa co najmniej $k$.

\subsection{Rozwiązanie}
Odpowiedź możemy wyszukać binarnie. Potrzebujemy funkcji, która sprawdzi, czy istnieje taki fragment długości $\geq k$, którego mediana jest $\geq X$, gdzie $X$ będzie wartością na której wyszukujemy binarnie.
\\ W naszej funkcji zaczniemy od nadania etykiety każdemu elementowi ciągu:
$$
\begin{cases}
	1 & \text{jeśli element} \geq X \\
	-1 & \text{w przeciwnym wypadku.}
\end{cases}
$$
Okazuje się, że na tak skonstruowanym ciągu, każdy fragment, którego suma jest liczbą dodatnią posiada medianę $\geq X$. Teraz użyjemy sum i minimów prefiksowych do znalezienia dowolnego fragmentu, którego suma jest dodatnia, a długość nie mniejsza niż $k$. Będziemy dla każdego prefiksu $i$ znajdować minimalny prefiks $j$ z przedziału $[1,i-k]$. Jeśli ich różnica jest liczbą dodatnią to przerywamy działanie funkcji i zwracamy prawdę.
\\ Ostatecznie odpowiedzią dla zadania będzie największa wartość $X$, dla której nasza funkcja zwraca prawdę. Złożoność: $O(n\log{}n)$

\end{document}